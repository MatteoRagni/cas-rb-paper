% !TEX root = Ragni2016.tex

\newcommand{\Ruby}{\emph{Ruby}}
\newcommand{\Mruby}{\emph{mRuby}}
\newcommand{\Jruby}{\emph{JRuby}}
\newcommand{\ragnicas}{\emph{ragni-cas}}

%% Library command
\newcommand{\Array}{\texttt{Ar\-ray}}
\newcommand{\Float}{\texttt{Flo\-at}}
\newcommand{\Fixnum}{\texttt{Fix\-num}}

\newcommand{\Namespace}{\texttt{::}}

\newcommand{\CAS}{\texttt{CAS}}
\newcommand{\CASOp}{\CAS\Namespace\texttt{Op\-}}
\newcommand{\CASBinaryOp}{\CAS\Namespace\texttt{Bi\-na\-ry\-Op}}
\newcommand{\CASNaryOp}{\CAS\Namespace\texttt{Nary\-Op}}

\newcommand{\CASExpression}{\CAS\Namespace\texttt{Con\-di\-tion}}

\newcommand{\CASSin}{\CAS\Namespace\texttt{Sin}}
\newcommand{\CASSum}{\CAS\Namespace\texttt{Sum}}
\newcommand{\CASDiff}{\CAS\Namespace\texttt{Diff}}

\newcommand{\CASVariable}{\CAS\Namespace\texttt{Va\-ria\-ble}}
\newcommand{\CASConstant}{\CAS\Namespace\texttt{Con\-stant}}
\newcommand{\CASFunction}{\CAS\Namespace\texttt{Func\-tion}}

\newcommand{\CASOpsimplify}{\CASOp\texttt{\#sim\-pli\-fy}}
\newcommand{\CASOpsubs}{\CASOp\texttt{\#subs}}
\newcommand{\CASOpdiff}{\CASOp\texttt{\#diff}}
\newcommand{\CASOpcall}{\CASOp\texttt{\#call}}
