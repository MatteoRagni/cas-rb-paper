% !TEX root = Ragni2016.tex

%  ___                     _
% |_ _|_ __  _ __  __ _ __| |_
%  | || '  \| '_ \/ _` / _|  _|
% |___|_|_|_| .__/\__,_\__|\__|
%           |_|
\section{Impact}
\label{sec:impact}

% \textbf{This is the main section of the article and the reviewers weight the description here appropriately}
% Indicate in what way new research questions can be pursued as a result of the software (if any).
% Indicate in what way, and to what extent, the pursuit of existing research questions is improved (if so).
% Indicate in what way the software has changed the daily practice of its users (if so).
% Indicate how widespread the use of the software is within and outside the intended user group.
% Indicate in what way the software is used in commercial settings and/or how it led to the creation
% of spin-off companies (if so).
There are different complete CAS systems on the market, with complete solutions for analysis of analitical models. But exporting a model, for optimization or any other research activity, requires a lot of work, even with a good CAS software.

This library is a midpoint between a CAS and an AD library. It allows to manipulate expressions while mantaining the complete control on how the code is exported. Each rule is overloaded and applied runtime, without the need of compilation. Each user's model may include the mathematical description, code generation rules and high level logic that should be intrisic to such a rule --- e.g.~exporting gradients as \textbf{patterns} instead of matrices.

Our research group is including \texttt{ragni-cas} in a solver for optimal control problem with indirect methods, as interface for problems' description~\cite{biral2016notes}.

As a long term ambitious impact, this library will become a complete CAS for \Ruby~language, filling the empty space reported by \emph{SciRuby} for symbolic math engines. This will require time, and the gem's MIT license allows everyone to contribute to the project.
