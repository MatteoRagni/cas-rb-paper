% !TEX root = Ragni2016.tex

%   ___             _         _
%  / __|___ _ _  __| |_  _ __(_)___ _ _  ___
% | (__/ _ \ ' \/ _| | || (_-< / _ \ ' \(_-<
%  \___\___/_||_\__|_|\_,_/__/_\___/_||_/__/
\section{Conclusions}
\label{sec:conclusions}

% Set out the conclusion of this original software publication.
This work presents a pure \Ruby~library that implements a minimalistics CAS with
automatic and symbolic differentiation that is aimed at code generation and metaprogramming.
Although at an early developing stage, \ragnicas~has promising feature, some of them
shown in Section~\ref{sec:examples}. Also, this is the only gem that implements
symbolic manipulation for this language.

Language features and lack of dependencies simplify the use of the module as interface, extending model definition
capabilities for numerical algorithms. All core functionalities and basic mathematics are defined, with the plan to include more features in next releases. Reopening a class guarantees a
\emph{liquid} behaviour, in which users are free to modify core methods and their needs.

Library is published in \emph{rubygems.org} repository and versioned on \emph{github.com}, under MIT license.
It can be included easily in projects and in inline interpreter, or installed as a standalone gem.
