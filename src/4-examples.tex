% !TEX root = Ragni2016.tex

%  ___                     _
% | __|_ ____ _ _ __  _ __| |___ ___
% | _|\ \ / _` | '  \| '_ \ / -_|_-<
% |___/_\_\__,_|_|_|_| .__/_\___/__/
%                    |_|
\section{Illustrative Examples}
\label{sec:examples}

\subsection{Code Generation as C Library}
In this example \review{it is shown how a \emph{user of \ragnicas} can export a mathematical model as a C library}. \texttt{c-opt} plugin implements advanced features such as code optimization and generation of libraries.

The library \texttt{example} implements the model:
\begin{equation}
f(x, y) = x^y + g(x)\, \log(\sin(x^y))
\end{equation}
Expression $g(x)$ belongs to a external object, declared as \texttt{g\_impl}, and its interface is described in \texttt{g\_impl.h} header. \review{What should be noted is the form of the code exported}: the intermediate operation $x^y$ is evaluated once, even if appears twice in our model. The C function that implements our model $f(x,y)$ is declared with the token \texttt{f\_impl}. The exporter uses as default type \texttt{double} for variables and function returned values.

\noindent%
\lstinputlisting[caption={Calling optimized-C exporter for library generation},label={code:example-exporting-C-1}]{./scripts/example_01.rb}

Library created by \texttt{CLib} contains the following code:

\noindent%
\begin{minipage}{.5\textwidth}
\lstinputlisting[style=customruby,language=C,caption={C Header}]{./scripts/source.h}
\end{minipage}\hfill
\begin{minipage}{.5\textwidth}
\lstinputlisting[style=customruby,language=C,caption={C Source},frame=tbl]{./scripts/source.c}
\end{minipage}

The function $g(x)$ models the following operation:
\begin{equation}
g(x) = (\sqrt{x + a} - \sqrt{x}) + \sqrt{\pi + x}
\end{equation}
and may suffer from \emph{catastrophic \review{numerical} cancellation}~\cite{higham2002accuracy} \review{when $x$ value is considerably greater than $a$}. User \review{may decide to} specialize code generation rules for this particular expression, \review{stabilizing it} through rationalization. \review{Without modifying the actual model} $g(x)$ in Listing~\ref{code:example-exporting-C-2} the rationalization is inserted \review{into exportation rules} for differences of square roots
\footnote{i.e.:
$\sqrt{\phi(\cdot)} - \sqrt{\psi(\cdot)} =
\dfrac{\phi(\cdot) - \psi(\cdot)}{\sqrt{\phi(\cdot)} + \sqrt{\psi(\cdot)}}$}.
\review{This rule is valid only for the current user script}. For more insight about \verb!__to_c! and \verb!__to_c_impl!\review{,} refer to the software manual.

\noindent%
\lstinputlisting[caption={Conditioning in exporting function},label={code:example-exporting-C-2}]{./scripts/example_02.rb}

It should be noted the \textbf{separation between the model}---that does not contain \review{stabilization}---\textbf{and the code generation rule}---that overloads, for this particular case and this particular language, the predefined code generation rule \review{to obtain the conditioned code}. Obviously, a user can decide to apply directly the conditioning on the model\review{, but this may change the calculus behavior in further manipulation}. The result of Listing~\ref{code:example-exporting-C-2} is reported:

\noindent%
\begin{minipage}{.5\textwidth}
\lstinputlisting[style=customruby,language=C,caption={\texttt{g\_impl} Header}]{./scripts/g_impl.h}
\end{minipage}\hfill
\begin{minipage}{.5\textwidth}
\lstinputlisting[style=customruby,language=C,caption={\texttt{g\_impl} Source},frame=tbl]{./scripts/g_impl.c}
\end{minipage}

\subsection{Using the module as interface}
As example, an implementation of an algorithm that estimates the \emph{order of convergence} for trapezoidal integration scheme~\cite{weideman2002numerical} is provided, using the symbolic differentiation as interface.

Given a function $f(x)$, the trapezoidal rule for primitive estimation in the interval $[a,\,b]$ is:
\begin{equation}
  I_{n}(a, b) = h\, \left( \dfrac{f(a) + f(b)}{2} +
    \sum\limits_{k = 1}^{n - 1}{f \left( a + k \,h \right)} \right)
\end{equation}
with $h = (b - a) / n $, where $n$ mediates the integration's step size. When exact primitive $F(x)$ is known, approximation error is:
\begin{equation}
  E[n] = F(b) - F(a) - I_{n}(a, b)
\end{equation}
The error has an asymptotic expansion of the form:
\begin{equation}
  E[n] \propto C\,{n}^{-p}
\end{equation}
where $p$ is the convergence order. Using a different value for $n$, for example $2\,n$, the ratio~\ref{eq:error.ratio} takes the approximate vale:
\begin{equation}
  \label{eq:error.ratio}
  \dfrac{E[n]}{E[2\,n]} \approx 2^{p} \quad \rightarrow \quad p \approx \log_2 \left( \dfrac{E[n]}{E[2\,n]} \right)
\end{equation}
Following Listings contain the implementation of the described procedure using the proposed gem and the well known \emph{Python}~\cite{van2011python} library \emph{SymPy}~\cite{christopher_smith_2016_47274}.

\noindent%
\begin{minipage}{.5\textwidth}
\lstinputlisting[style=customruby,caption={Ruby version}]{./scripts/example_03.rb}
\end{minipage}\hfill
\begin{minipage}{.5\textwidth}
\lstinputlisting[style=customruby,language=python,caption={Python version},frame=tbl]{./scripts/example_03.py}
\end{minipage}

\review{
\subsection{ODE Solver with Taylor's series}

In this example we assume a user needs to generate a solving step for specific ODE problems, using Taylor's series method~\cite{Butcher20081}. Given an ODE in the form:
\begin{equation}
\label{eq:taylor.ode}
y'(x) = f\left(x, y(x)\right)
\end{equation}
the integration step with order $n$ has the form:
\begin{equation}
y(x + h) = y(x)  + h\,y'(x) + \dots + \dfrac{h^{n}}{n!} \, y^{(n)}(x) + E_{n}(x)
\end{equation}
where, obviously, it is possible to use equation~\ref{eq:taylor.ode}, which brings to the following recurrent identity:
\begin{equation}
{y}^{(i)}(x) =
\dfrac{\partial {y}^{(i-1)}(x)}{\partial x} +
\dfrac{\partial {y}^{(i-1)}(x)}{\partial y} {y'}(x)
\end{equation}
For this algorithm, three methods are defined. The first evaluates the factorial, the second evaluates the list of required derivatives, and the third returns the integration step in a symbolic form. The result of the third method is transformed in a C function. In this particular case, the ODE is $y' = x y$.
\lstinputlisting[caption={Generator for ODE integration step},label={code:example-ode}]{./scripts/listing_07.rb}
For the resulting C code, please check online version of the examples.

Other usage examples are available online\footnote{\url{http://bit.ly/Mr_CAS_examples}}---i.e.~adding a user defined \CASOp that implements the $\mathrm{sign}(\cdot)$ function with the appropriate optimized C generation rule; exporting this operation as a continuous function through overloading or substitutions; performing a symbolic Taylor's series; writing an exporter for the \LaTex language.
} % review
