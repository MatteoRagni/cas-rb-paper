%  ___                     _
% | __|_ ____ _ _ __  _ __| |___ ___
% | _|\ \ / _` | '  \| '_ \ / -_|_-<
% |___/_\_\__,_|_|_|_| .__/_\___/__/
%                    |_|
\section{Illustrative Examples}
\label{sec:examples}

\subsection{Code Generation as C Library}
This example shows how to export a C library using the \texttt{CAS} module as design interface. \texttt{c-opt} plugin implements advanced features such as code optimization and generation of libraries.

In this example we create a library \texttt{example} that implements the model:

\begin{equation}
f(x, y) = x^y + g(x)\, \mathrm{log}(\sin(x^y))
\end{equation}

Expression $g(x)$ is implemented as \texttt{g\_impl} and its interface is described in the external header \texttt{g\_impl.h}. The code must be optimized: the intermediate operation $x^y$ should be evaluated once, even if required twice in our model. The C function that implements our model $f(x,y)$ should be called with the token \texttt{f\_impl}. The exporter uses as default type, for variables and function returned values, \texttt{double}.

\begin{lstlisting}[caption={Calling optimized-C exporter for library generation},label={code:example-exporting-C-1}]
require 'ragni-cas/c-opt'

# Models
x, y = CAS.vars :x, :y
g = CAS.declare :g, x

f = x ** y + g * CAS.log(CAS.sin(x ** y))

# Code Generation
g.c_name = 'g_impl'             # g token

CAS::CLib.create "example" do
  include_local "g_impl"        # g header
  implements_as "f_impl", f     # token for f
end
\end{lstlisting}
Library created by class \texttt{CLib} contains the following code:

\noindent%
  \begin{minipage}{.5\textwidth}
    \lstinputlisting[style=customruby,language=C,caption={C Header}]{./scripts/source.h}
  \end{minipage}\hfill
  \begin{minipage}{.5\textwidth}
    \lstinputlisting[style=customruby,language=C,caption={C Source},frame=tbl]{./scripts/source.c}
  \end{minipage}

\subsection{Using the module as interface}
As example, an implementation of an algorithm that extimates the \emph{order of convergence} for trapezoidal integration scheme \cite{weideman2002numerical} is provided, using the automatic differentiation as interface.

Given a function $f(x)$, the trapezoidal rule for primitive estimation in the interval $[a,\,b]$ is:
\begin{equation}
  I_{n}(a, b) = \dfrac{b - a}{n} \left( \dfrac{f(a) + f(b)}{2} +
    \sum\limits_{k = 1}^{n - 1}{f \left( a + k \dfrac{b - a}{n} \right)} \right)
\end{equation}
where $n$ mediates the integration's step size. When exact primitive $F(x)$ is known, approximation error is:
\begin{equation}
  E[n] = F(b) - F(a) - I_{n}(a, b)
\end{equation}
This error shows a direct relation:
\begin{equation}
  E[n] \propto C\,{n}^{-p}
\end{equation}
where $p$ is the convergence order. Using a different value for $n$, for example $2\,n$:
\begin{equation}
  \dfrac{E[n]}{E[2\,n]} \approx 2^{p} \quad \rightarrow \quad p \approx log_2 \left( \dfrac{E[n]}{E[2\,n]} \right)
\end{equation}
Following listings contain the implementation of the described procedure using the described gem and the well known \emph{Python} \cite{van2011python} library \emph{sympy} \cite{christopher_smith_2016_47274}.

\noindent%
  \begin{minipage}{.5\textwidth}
    \lstinputlisting[style=customruby,caption={Ruby version}]{./scripts/test.rb}
  \end{minipage}\hfill
  \begin{minipage}{.5\textwidth}
    \lstinputlisting[style=customruby,language=python,caption={Python version},frame=tbl]{./scripts/test.py}
  \end{minipage}
