%  __  __     _   _          _   _
% |  \/  |___| |_(_)_ ____ _| |_(_)___ _ _
% | |\/| / _ \  _| \ V / _` |  _| / _ \ ' \
% |_|  |_\___/\__|_|\_/\__,_|\__|_\___/_||_|
\section{Motivation and significance}
\label{sec:motivation}

% Introduce the scientific background and the motivation for developing the software.
% Explain why the software is important, and describe the exact (scientific) problem(s) it solves.
% Indicate in what way the software has contributed (or how it will contribute in the future) to the
% process of scientific discovery; if available, this is to be supported by citing a research paper
% using the software.
% Provide a description of the experimental setting (how does the user use the software?).
% Introduce related work in literature (cite or list algorithms used, other software etc.).
\Ruby~is a purely object-oriented scripting language that allows to express several programming paradigms. It was designed in the mid-1990s by Yukihiro Matsumoto (also known as \emph{Matz}), and it is internationally standardized since 2012 as ISO/IEC 30170.

With the advent of the \emph{Internet of Things}, a written from scratch version of the \Ruby~interpreter called \Mruby~(\emph{eMbedded Ruby}) has been published on \emph{GitHub} by Matsumoto in 2014. The new interpreter is a lightweight implementation aimed at both low power devices and personal computer that complies with the standard. \Mruby~has a completely new API, and it is designed to be embedded in a complex project as a front-end interface --- e.g.\ a numerical optimization suite may use \Mruby~to get problem input definitions.

The \Ruby~code-base exposes a a large set of utilities in core and standard library. This set of tool can be furthermore expanded through libraries, also known as \emph{gems}. Even the high number of gems deployed and available, there is no complete library that implements a \textbf{symbolic automatic differentiation} (AD) engine that also handles some basic computer algebra routines that is cross compatible with all the different \Ruby~interpreter.

\Ruby~has matured its fame as a web oriented language because of its capabilities web templete system and in general for processing complex content. This characteristic allows to efficiently generate code in other languages, and given such a gem, it is really easy to develop rapidly a specific code generator for well known software --- e.g.\ IPOPT\@.

The library that is described in this work, is a gem implemented in pure \Ruby~code and thus compatible with all interpreter that complies with the standard, that is able to perform symbolic AD and some simple computer algebra operations. In particular the library aims at:
\begin{itemize}
  \item be an instrument for rapid development of prototype interface for numerical algorithms --- including the \Mruby~engine or exporting generated code --- that can be in different languages;
  \item the library allows to rapidly test workaround for numerical issues particular to certain problems, by changing on request how the code is exported, and how basic functions are formulated;
  \item long term impact, quite ambitious, of creating a complete open-source CAS system for the \Ruby~language, that must be compatible with all the interpreters that comply with the standard.
\end{itemize}

This is not the first gem that is able to perform AD. The available computer algebra library for \Ruby~are:
\begin{description}
  \item [\emph{Rucas}] gem at early stage and with discontinued developing status; it implements basic simplification routines. There is no AD method, but it is one of the milestones.
  \item [\emph{Symengine}] is a wrapper for the C++ library \emph{symengine}. The back-end library is very complete, but it is compatible only with the mainstream \Ruby~interpreter. At the moment, the \emph{SciRuby} project reports the gem as broken, and removed it from its codebase. From a direct test, when performing AD of a function, the engine returns always \texttt{nil}.
  \item [\emph{Datamelt}] probably the most complete choice, is a Java library and compatible with one particular \Ruby~virtual machine, called \Jruby.
\end{description}
